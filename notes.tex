%! Author = Kevin Lin
%! Date = 9/30/2025

% Preamble
\documentclass[11pt,a4paper,margin=1in]{article}

% Packages
\usepackage{amsmath}
\usepackage{amssymb}
\usepackage{enumerate}

\title{Notes}
\author{Kevin Lin}
\date{9/30/2025}

% Document
\begin{document}
\maketitle
\section{}
\begin{gather*}
\forall A \in \mathbb{R}\\
\frac{\frac{1}{8}}{\frac{1}{2} + \frac{1}{8}}\\
\dfrac{1}{2}\\
\Omega\\
\omega\\
\end{gather*}
\begin{enumerate}[A.]
    \item A
    \item B
\end{enumerate}

\section{Extended Monty Hall Problem}
\begin{flushleft}
    Suppose you have $n$ doors, where behind 1 is the car and behind $n - 1$ are
    goats. Monty Hall will open $k ([0, n - 2])$ doors as he has to leave one door
    unopened for you to switch to, and the original door you picked. The chance
    that you picked the car originally is $\frac{1}{n}$, hence the chance you didn't
    pick the car is $\frac{n - 1}{n}$. When Monty Hall opens $k$ doors, the probability
    that you should switch to win is now determined by the probability you didn't 
    pick the correct door the first times multiplied by the new probability that
    you pick the correct door when you switch, given by:
\end{flushleft}
\begin{gather*}
    \frac{n - 1}{n} \cdot \frac{1}{n - k - 1} = \frac{1}{n} \cdot \frac{n - 1}{n - k - 1}
\end{gather*}
\begin{flushleft}
    Suppose Monty Hall opens $n - 2$ doors, then when you switch the probablity
    of winning becomes apparent:
\end{flushleft}
\begin{gather*}
    \frac{n - 1}{n} \cdot \frac{1}{n - (n - 2) - 1} = \frac{n - 1}{n}
\end{gather*}
\begin{flushleft}
    In the standard Monty Hall problem where $n = 3$ and $k = 1$, the probability 
    of winning when you switch is:
\end{flushleft}
\begin{gather*}
    \frac{3 - 1}{3} \cdot \frac{1}{3 - 1 - 1} = \frac{2}{3}
\end{gather*}
\end{document}